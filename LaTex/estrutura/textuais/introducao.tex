% INTRODUÇÃO-------------------------------------------------------------------

\chapter{INTRODUÇÃO}
\label{chap:introducao}

Acidentes de trabalho causam problemas incalculáveis tanto para o trabalhador, quanto para a empresa que possui perdas de ativos tangíveis, intangíveis e capital humano. O Brasil tem sido um dos países que mais sofrem com esse importante problema, onde ocorreu, neste século, um acidente de trabalho fatal a cada duas horas e meia, conforme Mattos (2011).  O autor complementa que nos países em desenvolvimento 10 do produto interno bruto (PIB) é perdido devido a doenças e agravos ocupacionais que além da morte e de sofrimento para o trabalhador e para sua família, problemas ainda pouco estudados, os acidentes de trabalho têm reflexos socioambientais, econômicos e políticos para toda a sociedade.
O Tribunal de Justiça do Trabalho, define acidente de trabalho, no artigo 19 da Lei nº 8.213 de 1991, como:

\begin{citacao}
Acidente de trabalho é o que ocorre pelo exercício do trabalho a serviço da empresa ou pelo exercício do trabalho dos segurados referidos no inciso VII do art. 11 desta lei, provocando lesão corporal ou perturbação funcional que cause a morte ou a perda ou redução, permanente ou temporária, da capacidade para o trabalho. Ao lado da conceituação acima, acedente de trabalho típico, por expressa determinação legas, as doenças profissionais e/ou ocupacionais equiparam-se a acidentes de trabalho. (TRIBUNAL DE JUSTIÇA DO TRABALHO, 2020). 
\end{citacao}

Há muito tempo já existia relatos de doenças relacionadas as atividades laborais, pelos Egípcios, Babilônicos e greco-romanos. Há relatos de registros Egípcios relacionando acidentes de trabalho e os riscos inerentes às atividades datadas de 2360 a.C, primeiramente, relacionado a uma premissa mágico-religiosa e, posteriormente, numa perspectiva naturalista (MATTOS, 2011).
Mostrando que são problemas antigos e por muitas vezes inerentes à atividade exercida pelo indivíduo, Mattos (2011) contextualiza alguns acontecimentos relatados pelo povo Grego, sendo eles os responsáveis pela mudança do paradigma espiritualista para o naturalista. Como exemplo, Hipócrates descreveu a intoxicação saturnina, porém sem descrever o ambiente e a ocupação laboral. Plínio, o Velho (23-79 a.C.), relatou em seu trabalho, o tratado História Naturalis, o aspecto de trabalhadores expostos ao chumbo, mercúrio e poeiras, descrevendo os primeiros equipamentos de proteção individual (EPIs), sendo estas máscaras feitas de pano e bexigas.
Os autores Barsano e Barbosa (2012) descrevem algumas premissas das condições de trabalho de forma evitar acidentes e otimizar o nível de segurança dos trabalhadores:

\begin{citacao}
Portanto, no ambiente de trabalho necessitamos encontrar condições capazes de proporcionar o máximo de proteção e ao mesmo tempo satisfação no trabalho. E quando existe essa combinação, com toda certeza resulta num aumento significativo da produtividade, melhoria da qualidade dos serviços, redução do índice de absenteísmo e diminuição drástica das doenças e dos acidentes do trabalho. (BARSANO; BARBOSA, 2012). 
\end{citacao}

\section{TEMAS DE PESQUISA}
\label{sec:temasDePesquisa}

Mensurar e supervisionar a exposição a ambientes insalubres conforme a Norma Regulamentadora (NR) 15, Atividades e Operações Insalubres (MINISTÉRIO DO TRABALHO E EMPREGO, 2007), pois segundo Barsano e Barbosa (2012): quando um desses fatores, ou um conjunto deles, foge ao controle, seja pelos níveis permitidos ou pelos processos que se desencadeiam, o ambiente de trabalho torna-se suscetível de desenvolver as chamadas patologias do trabalho. Estes agentes são descritos na NR 9 (MINISTÉRIO DO TRABALHO E EMPREGO, 1978) em três categorias: físicos, químicos e biológicos. 

\section{DELIMITAÇÃO DE PESQUISA}
\label{sec:delimitacaoDePesquisa}

Nesta pesquisa, alguns agentes físicos serão avaliados, sendo que a NR 9 descreve os agentes físicos, como: ruídos, vibrações, pressões anormais, temperaturas extremas, radiações ionizantes, radiações não ionizantes, bem como o infrassom e ultrassom.
Os agentes físicos que estão em foco deste Trabalho de Conclusão de Curso (TCC) são: ruídos, vibrações, temperaturas e radiações não ionizantes (luz), que podem ser facilmente mensurados e comparados com os valores estabelecidos na norma NR 15, de forma a ter um melhor controle sob a exposição e prevenido as patologias, sendo mensurados por sensores de baixa complexidade e podendo ser embarcado em dispositivo vestível de baixo custo.

\section{PROBLEMÁTICA E PREMISSAS DE PESQUISA}
\label{sec:problematicaEPremissasDePesquisa}

De acordo com Lusk et al. (apud MEIRA et al, 2012): “O ruído causa vários efeitos indesejáveis à saúde dos indivíduos expostos, como zumbido, aumento da pressão arterial e da frequência cardíaca, insônia, estresse e irritabilidade”.
A exposição à vibração ocupacional, gera diversos danos à saúde do trabalhador como doenças vasculares, neurológicas e musculares. Cristiano Molica (2018), em entrevista ao PodPrevenir, descreve que “uma vibração de alta amplitude nas mãos por exemplo, gera uma síndrome chamada síndrome do dedo branco que afeta todo o sistema circulatório podendo até gerar amputação do membro superior”. 
A exposição ao calor causa brotoejas, câimbras, exaustão e insolação; com os sintomas de que o corpo precisa ser resfriado sendo: dor de cabeça, fraqueza, suor excessivo, irritabilidade ou confusão mental, sede, náuseas e vômitos diz a fabricante de EPIs Dupont (2019) em seu blog.
Lorenzi (2020) lista as doenças causadas pela má iluminação, sendo elas: irritação dos olhos, cansaço visual, distúrbios emocionais e problemas de pele.
Os ambientes de trabalho são lugares dinâmicos e muito complexos. Apenas os fatores descritos acima já influenciam uma quantidade considerável de pessoas nas áreas de manufatura de produtos e beneficiamento de alimentos, saúde, escritórios e outros tipos de serviços. Sendo assim, as normas e procedimentos atuais realmente são suficientes para garantir que todos esses colaboradores estão protegidos dessas enfermidades?
A mensuração constante, controle automático e supervisão desses parâmetros mesmo que não seja feita com instrumentos precisos desenvolvidos para tal, podem indicar a magnitude e dar uma ideia de pontos a serem observados com mais cuidado.

\section{OBJETIVOS DA PESQUISA}
\label{sec: objetivosDaPesquisa}

\subsection{Objetivo Geral}
\label{sec: objetivoGeral}

Desenvolver um sistema embarcado composto de um dispositivo vestível com sensores de temperatura, vibração e ruído com capacidade de transferência de dados para um servidor WEB de supervisão e monitoramento. 

\subsection{Objetivos Específicos}
\label{sec: objetivoEspecificos}

\begin{itemize}
    \item   Selecionar os sensores.
    \item	Integrar os sensores ao sistema microcontrolador.
    \item	Desenvolver o circuito embarcado.
    \item	Desenvolver o firmware.
    \item	Desenvolver o servidor.
    \item	Desenvolver a aplicação WEB.
    \item	Integrar o dispositivo e a aplicação ao servidor.

\end{itemize}

\section{JUSTIFICATIVA}
\label{sec: justificativa}

Um acidente não ocorre por uma única causa, mas sim por uma série de fatores contribuintes.

\begin{citacao}
A abordagem holística da segurança do trabalho é outra forma em que visualizamos os acidentes. Nela não afirmamos que o acidente teve uma única e exclusiva origem, mas foi gerado pela interação simultânea de diversos fatores (físicos, biológicos, psicológicos, sociais e culturais), e que um desencadeou o outro, gerando um acidente. Logo não há uma causa única dos acidentes, e sim várias. (BARSANO e BARBOSA, 2018, p. 24). 
\end{citacao}

Lito Sousa (2015) destaca que as capacidades humanas se deterioram com a complacência e que cada vez mais a ciência e a tecnologia trabalham para que os erros diminuam e na impossibilidade de eliminá-los, que seus efeitos não causem um acidente grave ou fatal.
A aviação trabalha a partir do fato de que seres humanos são falhos e por isso dispõem de diversas camadas de proteção contra acidentes. Souza (2009) expõem que por vezes a mente pode criar uma falsa memória baseada em experiências anteriores, ou seja, um procedimento de segurança pode ser esquecido pois o cérebro pode ter injetado a memória de ter executado aquela tarefa a partir da memória de centenas de execuções posteriores.
Dado os altos números de acidentes destacado no início deste trabalho e as limitações do ser humano, surge a percepção inspirada na aviação da necessidade de mais camadas de segurança para combater os altos números supra citados, uma camada com base na tecnologia que reduza os erros decorrentes do homem, colete dados, tome ações de controle e supervisione o dia a dia do trabalhador. 

\section{PROCEDIMENTOS METODOLÓGICOS}
\label{sec: procedimentosMetodologicos}

O trabalho inicia-se com a consulta à NR 9 sobre os limites de exposição aos agentes físicos. Desta forma, poder-se-á ter uma noção da escala, acurácia e precisão dos sensores. Assim, a próxima etapa é a escolha dos sensores com base nos requisitos e facilidade de implementação.
Dando início ao processo de prototipagem do dispositivo de forma que seja vestível e tenha embarcado cada um dos sensores, será realizado o desenvolvimento do firmware para a correta leitura das grandezas a serem medidas, transmissão destes dados ao servidor e emissão de alertas ao usuário.
O próximo passo é o desenvolvimento do servidor que receberá os dados e registra os avisos de forma que seja visível aos gestores. Seguido dos testes de hardware, firmware e software. 
Por fim, serão obtidos os resultados para a última etapa, na qual será feita a análise do sistema para se concluir se o dispositivo obteve êxito e também o registro dos principais pontos de aprendizado.


\section{ESTRUTURA}
\label{sec: estrutura}

Este trabalho será composto por seis capítulos: o primeiro será uma abordagem do tema, sua delimitação, as problemáticas, objetivos, justificativa e descrição dos procedimentos metodológicos adotados. No segundo capítulo será feita uma revisão bibliográfica. O terceiro é apresentado a proposta do sistema e do dispositivo. No quarto será discutido os detalhes de implementação e as tecnologias dos sensores, acumuladores de energia, comunicação, sistemas microcontrolados. Também será discutida as tecnologias empregadas no servidor e na aplicação WEB. No quinto capítulo será discutido o desenvolvimento e implementação do sistema. O último será discutido os resultados obtidos a partir de testes em ambiente controlado do dispositivo. 

\section{CRONOGRAMA}
\label{sec: cronograma}

O trabalho será desenvolvido em um período de três semestres letivos com cada etapa sendo detalhada na...

